
Existen proyectos similares que se han realizado anteriormente:

\subsection{Diseño de un Contenedor Municipal Automático}
“DISEÑO DE UN CONTENEDOR MUNICIPAL AUTOMÁTICO MEDIANTE RED DE SENSORES CON MONITOREO INALÁMBRICO PARA LA MEJORA DEL PROCESO DE SELECCIÓN DE RESIDUOS SÓLIDOS, CALLAO 2022”. Este trabajo de investigación tiene como fin plantear una solución a una realidad que afecta a la sociedad actual, específicamente en el entorno urbano, donde la población citadina crece año tras año debido al centralismo. Este crecimiento poblacional conlleva un aumento en la producción de residuos de diversas categorías, entre ellos, los residuos sólidos. 

Como objetivo general, tiene diseñar un contenedor municipal automático mediante red de sensores y monitoreo inalámbrico, capaz de mejorar el proceso de selección de residuos sólidos. El sistema utiliza sensores y un microcontrolador para distinguir entre diferentes tipos de desechos, como plástico, vidrio, cartón y papel. El resultado final de este proyecto es un prototipo de un contenedor de residuos sólidos inteligente, que tiene instalado un módulo GPS para controlar la ubicación y un sensor ultrasónico que mide el nivel de llenado. En la Figura 90 se muestra el diseño del dispositivo que será ubicado dentro de un contenedor de residuos sólidos. Este dispositivo se compone de un sensor ultrasónico, un módulo GPS, un módulo ESP32 con su batería recargable, un transceptor LoRa RFM95 con una antena helicoidal y un panel solar para cargar las baterías de litio \cite{manrique}.

\subsection{Sistema de Clasificación Inteligente de Residuos}
“INTELLIGENT WASTE SORTING SYSTEM: LEVERAGING ARDUINO FOR AUTOMATED TRASH IDENTIFICATION AND CATEGORIZATION”. En este trabajo de investigación, se utiliza un conjunto de datos personalizado, una colección completa de imágenes de categorías de desechos divididas en siete clases, con modelos de detección de objetos como YOLO para automatizar la detección de basura y la clasificación precisa, mejorando así la precisión en la clasificación de desechos. El objetivo principal es aprovechar los atributos distintivos de los modelos YOLO combinados con Arduino para desarrollar un sistema de clasificación de basura más eficaz. Los procesos tradicionales de gestión de basura son laboriosos, lo que los hace lentos, costosos y propensos a errores humanos. Para superar estas deficiencias, se propone un modelo basado en el aprendizaje profundo que pueda detectar y clasificar rápidamente varios tipos de desechos en tiempo real. YOLO, un algoritmo de detección de objetos de última generación, sirve como columna vertebral de este modelo, asegurando una identificación rápida y precisa de los desechos. La integración de un microcontrolador Arduino agrega una dimensión práctica e interactiva, facilitando la comunicación entre el modelo de aprendizaje profundo y el mundo físico, permitiendo que el sistema active acciones como la clasificación, el reciclaje o la alerta a las autoridades \cite{ali}.

\subsection{Contenedor de Reciclaje Automatizado de Bajo Costo}
“A LOW-COST AUTOMATED SORTING RECYCLE BIN POWERED BY ARDUINO MICROCONTROLLER”. Este artículo presenta el desarrollo de un contenedor de reciclaje de bajo costo que clasifica automáticamente diferentes tipos de desechos reciclables utilizando un microcontrolador Arduino. Los objetivos de este trabajo son construir un prototipo de contenedor de reciclaje con un mecanismo de detección que pueda clasificar los desechos reciclables (como metal, papel y plástico) y asignar automáticamente los desechos a particiones específicas del contenedor según sus tipos. Este artículo también presenta el análisis de sensibilidad del prototipo de contenedor de reciclaje para clasificar los desechos correctamente. La construcción del prototipo se divide en dos partes: i) detección y ii) mecánica. La parte de detección identifica el tipo de material de desecho, utilizando un sensor de proximidad inductivo para detectar metales y un diodo emisor de luz (LED) junto con una resistencia dependiente de la luz (LDR) para papel y plástico. La parte mecánica incorpora un servomotor junto con el microcontrolador para clasificar el tipo de desecho. Los resultados muestran que el prototipo es capaz de clasificar los residuos con éxito, especialmente los plásticos, aunque se necesita mejorar la sensibilidad a los residuos de papel y metal para lograr una segregación eficaz \cite{hassan}.
