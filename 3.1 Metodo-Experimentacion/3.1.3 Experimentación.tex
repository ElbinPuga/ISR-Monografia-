

\subsubsection{Desarrollo del Prototipo}

\textbf{Estructura Física:}
\begin{itemize}
    \item \textbf{Diseño de la estructura:} Nos proponemos diseñar una estructura robusta que soporte los contenedores y los componentes electrónicos. 
    La estructura debe ser de fácil acceso para los usuarios, permitiendo que los materiales sean presentados a la cámara.

    \item \textbf{Ubicación de la Cámara:} Esta parte se trata de instalar el ESP32 CAM en una posición elevada, asegurando que tenga un campo de visión claro para capturar el material que la persona va a presentar.
\end{itemize}


\subsubsection{Integración Electrónica:}
\begin{itemize}
    \item \textbf{Conexiones:} Conectamos la ESP32 CAM a una fuente de alimentación adecuada y a los servomotores con la principal tarea de abrir las compuertas.
    \item \textbf{Sensores ultrasonicos:} Usamos sensores ultrasónicos para detectar cuando el usuario presenta un material. Esto nos ayuda a que la cámara esté activa y lista para capturar la imagen del material.
\end{itemize}


\subsubsection{Programación del Sistema}

\textbf{Desarrollo del Algoritmo de Reconocimiento:}
\begin{itemize}
    \item \textbf{Recopilación de Datos:}
    \begin{itemize}
        \item Se trata de recopilar un DataSet de Kaggle donde se tengan imágenes de cartón, papel y plástico en diversas condiciones de iluminación y ángulos. Esto nos da la oportunidad de que el modelo sea robusto y pueda manejar variaciones en el entorno real.
        \item Luego, etiquetaremos las imágenes para crear un conjunto de datos que se utilizará en el entrenamiento del modelo de reconocimiento.
    \end{itemize}
    
    \item \textbf{Utilizar Trasnfer Learning:}
    \begin{itemize}
        \item Tomaremos una cnn pre-entrenada y aplicaremos técnicas de visión por computadora con el conjunto de datos recopilado. Este proceso puede incluir la validación cruzada para asegurar que el modelo no esté sobreajustado.
        \item Realizaremos pruebas de precisión del modelo con un conjunto de imágenes no utilizadas durante el entrenamiento para evaluar su rendimiento y comportamiento.
    \end{itemize}
    
    \item \textbf{Control del Hardware:}
    \begin{itemize}
        \item \textbf{Lógica de Control:}
        \begin{itemize}
            \item Se trata de programar la lógica que controlará la activación del reconocimiento de imágenes cuando el sensor ultrasonico detecte que un usuario está presentando un material.
            \item Una vez que se identifica el material, se activa la compuerta correspondiente, utilizando los servomotores.
        \end{itemize}
        
        \item \textbf{Sistema de Retroalimentación:}
        \begin{itemize}
            \item Implementaremos un sistema de retroalimentación visual (por ejemplo, luces LED) que indique al usuario si la clasificación fue correcta o incorrecta. Si el material es reconocido correctamente, se abrirá la compuerta; si no, se podrá mostrar un mensaje en una pantalla o con luces que indiquen que el material no fue clasificado.
        \end{itemize}
    \end{itemize}
\end{itemize}

\subsubsection{Interacción del Usuario}

\textbf{Proceso de Uso:}
\begin{itemize}
    \item El usuario se acercará al basurero inteligente y presentará el material al módulo ESP32 CAM.
    \item El sistema activará la cámara para tomar una imagen del material.
    \item El algoritmo analizará la imagen y, tras identificar correctamente el material, la compuerta correspondiente se abrirá para que el usuario deposite el residuo.
    \item En caso de que el material no sea reconocido, se proporcionará retroalimentación al usuario sobre la incorrecta clasificación con el uso del Buzzer.
\end{itemize}
