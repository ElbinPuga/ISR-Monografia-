Con el prototipo desarrollado esperamos obtener los siguientes datos:

\begin{enumerate}
    \item \textbf{Cantidad Total de Residuos:} Registrar el volumen total de residuos que se depositan en el basurero inteligente.
    \item \textbf{Cantidad de Residuos Clasificados Correctamente:} Contar cuántos residuos fueron identificados y clasificados correctamente por el sistema en comparación con los incorrectamente clasificados.
\end{enumerate}

\textbf{Generar gráficos para la precisión del algoritmo de clasificación}

Esta recolección de data nos permite los siguientes puntos:

\begin{enumerate}
    \item \textbf{Visualizar de forma clara los datos}
    \begin{enumerate}
        \item \textbf{La Comprensión}: Los gráficos transforman datos numéricos en visualizaciones fáciles de interpretar, lo que permite comprender rápidamente el rendimiento del algoritmo sin necesidad de analizar tablas complejas.
        \item \textbf{Identificar tendencias}: A través de gráficos de líneas podremos observar cómo la precisión ha cambiado a lo largo del tiempo, lo que puede indicar si el sistema está mejorando con el uso.
    \end{enumerate}
    
    \item \textbf{Comparar}
    \begin{enumerate}
        \item \textbf{Comparación entre categorías}: Con gráficos de barras, podemos comparar la precisión entre diferentes categorías de materiales, lo que ayuda a identificar cuál material se clasifica mejor y cuál necesita mejoras.
        \item \textbf{Evaluación de resultados}: Permiten comparar resultados de diferentes períodos de prueba o diferentes versiones del algoritmo, facilitando la toma de decisiones sobre ajustes y mejoras.
    \end{enumerate}

    \item \textbf{Identificar problemas}
    \begin{enumerate}
        \item \textbf{Detectar anomalías}: Un gráfico que muestra una caída considerable o preocupante para nosotros en la precisión puede alertar sobre problemas en el algoritmo o cambios en el comportamiento del usuario.
        \item \textbf{Clasificación Incorrecta}: Un gráfico de pastel puede muestrar la proporción de clasificaciones correctas frente a incorrectas puede ayudar a identificar el alcance del problema de clasificación.
    \end{enumerate}

    \item \textbf{Datos para historial de mejoras}
    \begin{enumerate}
        \item \textbf{Cronograma de Mejoras}: Utilizar gráficos para establecer un cronograma de versiones del algoritmo permite visualizar cuándo se implementarán mejoras y ajustes basados en los datos recolectados.
        \item \textbf{Documentación de Cambios}: Cada versión puede ir acompañada de gráficos que muestren la evolución de la precisión, lo que facilita la comunicación sobre el impacto de cada actualización en el rendimiento del sistema.
    \end{enumerate}

\end{enumerate}
