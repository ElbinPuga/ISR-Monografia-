El prototipo se puede aplicar en el centro regional de Veraguas de la Universidad Tecnológica de Panamá, específicamente en áreas de alta concurrencia como la cafetería y pasillos. Este entorno es ideal debido al volumen de desechos generados diariamente, y se espera que el prototipo contribuya a mejorar la tasa de reciclaje.

El impacto esperado incluye la reducción de residuos mezclados y una mayor concientización sobre la importancia de la correcta clasificación de desechos en el campus universitario. Además, se espera que el sistema genere datos útiles sobre los tipos de residuos más comunes en estas áreas.
